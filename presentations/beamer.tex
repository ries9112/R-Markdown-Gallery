% Options for packages loaded elsewhere
\PassOptionsToPackage{unicode}{hyperref}
\PassOptionsToPackage{hyphens}{url}
%
\documentclass[
  ignorenonframetext,
]{beamer}
\usepackage{pgfpages}
\setbeamertemplate{caption}[numbered]
\setbeamertemplate{caption label separator}{: }
\setbeamercolor{caption name}{fg=normal text.fg}
\beamertemplatenavigationsymbolsempty
% Prevent slide breaks in the middle of a paragraph
\widowpenalties 1 10000
\raggedbottom
\setbeamertemplate{part page}{
  \centering
  \begin{beamercolorbox}[sep=16pt,center]{part title}
    \usebeamerfont{part title}\insertpart\par
  \end{beamercolorbox}
}
\setbeamertemplate{section page}{
  \centering
  \begin{beamercolorbox}[sep=12pt,center]{part title}
    \usebeamerfont{section title}\insertsection\par
  \end{beamercolorbox}
}
\setbeamertemplate{subsection page}{
  \centering
  \begin{beamercolorbox}[sep=8pt,center]{part title}
    \usebeamerfont{subsection title}\insertsubsection\par
  \end{beamercolorbox}
}
\AtBeginPart{
  \frame{\partpage}
}
\AtBeginSection{
  \ifbibliography
  \else
    \frame{\sectionpage}
  \fi
}
\AtBeginSubsection{
  \frame{\subsectionpage}
}
\usepackage{lmodern}
\usepackage{amsmath}
\usepackage{ifxetex,ifluatex}
\ifnum 0\ifxetex 1\fi\ifluatex 1\fi=0 % if pdftex
  \usepackage[T1]{fontenc}
  \usepackage[utf8]{inputenc}
  \usepackage{textcomp} % provide euro and other symbols
  \usepackage{amssymb}
\else % if luatex or xetex
  \usepackage{unicode-math}
  \defaultfontfeatures{Scale=MatchLowercase}
  \defaultfontfeatures[\rmfamily]{Ligatures=TeX,Scale=1}
\fi
% Use upquote if available, for straight quotes in verbatim environments
\IfFileExists{upquote.sty}{\usepackage{upquote}}{}
\IfFileExists{microtype.sty}{% use microtype if available
  \usepackage[]{microtype}
  \UseMicrotypeSet[protrusion]{basicmath} % disable protrusion for tt fonts
}{}
\makeatletter
\@ifundefined{KOMAClassName}{% if non-KOMA class
  \IfFileExists{parskip.sty}{%
    \usepackage{parskip}
  }{% else
    \setlength{\parindent}{0pt}
    \setlength{\parskip}{6pt plus 2pt minus 1pt}}
}{% if KOMA class
  \KOMAoptions{parskip=half}}
\makeatother
\usepackage{xcolor}
\IfFileExists{xurl.sty}{\usepackage{xurl}}{} % add URL line breaks if available
\IfFileExists{bookmark.sty}{\usepackage{bookmark}}{\usepackage{hyperref}}
\hypersetup{
  pdftitle={Example Presentation},
  hidelinks,
  pdfcreator={LaTeX via pandoc}}
\urlstyle{same} % disable monospaced font for URLs
\newif\ifbibliography
\usepackage{color}
\usepackage{fancyvrb}
\newcommand{\VerbBar}{|}
\newcommand{\VERB}{\Verb[commandchars=\\\{\}]}
\DefineVerbatimEnvironment{Highlighting}{Verbatim}{commandchars=\\\{\}}
% Add ',fontsize=\small' for more characters per line
\usepackage{framed}
\definecolor{shadecolor}{RGB}{248,248,248}
\newenvironment{Shaded}{\begin{snugshade}}{\end{snugshade}}
\newcommand{\AlertTok}[1]{\textcolor[rgb]{0.94,0.16,0.16}{#1}}
\newcommand{\AnnotationTok}[1]{\textcolor[rgb]{0.56,0.35,0.01}{\textbf{\textit{#1}}}}
\newcommand{\AttributeTok}[1]{\textcolor[rgb]{0.77,0.63,0.00}{#1}}
\newcommand{\BaseNTok}[1]{\textcolor[rgb]{0.00,0.00,0.81}{#1}}
\newcommand{\BuiltInTok}[1]{#1}
\newcommand{\CharTok}[1]{\textcolor[rgb]{0.31,0.60,0.02}{#1}}
\newcommand{\CommentTok}[1]{\textcolor[rgb]{0.56,0.35,0.01}{\textit{#1}}}
\newcommand{\CommentVarTok}[1]{\textcolor[rgb]{0.56,0.35,0.01}{\textbf{\textit{#1}}}}
\newcommand{\ConstantTok}[1]{\textcolor[rgb]{0.00,0.00,0.00}{#1}}
\newcommand{\ControlFlowTok}[1]{\textcolor[rgb]{0.13,0.29,0.53}{\textbf{#1}}}
\newcommand{\DataTypeTok}[1]{\textcolor[rgb]{0.13,0.29,0.53}{#1}}
\newcommand{\DecValTok}[1]{\textcolor[rgb]{0.00,0.00,0.81}{#1}}
\newcommand{\DocumentationTok}[1]{\textcolor[rgb]{0.56,0.35,0.01}{\textbf{\textit{#1}}}}
\newcommand{\ErrorTok}[1]{\textcolor[rgb]{0.64,0.00,0.00}{\textbf{#1}}}
\newcommand{\ExtensionTok}[1]{#1}
\newcommand{\FloatTok}[1]{\textcolor[rgb]{0.00,0.00,0.81}{#1}}
\newcommand{\FunctionTok}[1]{\textcolor[rgb]{0.00,0.00,0.00}{#1}}
\newcommand{\ImportTok}[1]{#1}
\newcommand{\InformationTok}[1]{\textcolor[rgb]{0.56,0.35,0.01}{\textbf{\textit{#1}}}}
\newcommand{\KeywordTok}[1]{\textcolor[rgb]{0.13,0.29,0.53}{\textbf{#1}}}
\newcommand{\NormalTok}[1]{#1}
\newcommand{\OperatorTok}[1]{\textcolor[rgb]{0.81,0.36,0.00}{\textbf{#1}}}
\newcommand{\OtherTok}[1]{\textcolor[rgb]{0.56,0.35,0.01}{#1}}
\newcommand{\PreprocessorTok}[1]{\textcolor[rgb]{0.56,0.35,0.01}{\textit{#1}}}
\newcommand{\RegionMarkerTok}[1]{#1}
\newcommand{\SpecialCharTok}[1]{\textcolor[rgb]{0.00,0.00,0.00}{#1}}
\newcommand{\SpecialStringTok}[1]{\textcolor[rgb]{0.31,0.60,0.02}{#1}}
\newcommand{\StringTok}[1]{\textcolor[rgb]{0.31,0.60,0.02}{#1}}
\newcommand{\VariableTok}[1]{\textcolor[rgb]{0.00,0.00,0.00}{#1}}
\newcommand{\VerbatimStringTok}[1]{\textcolor[rgb]{0.31,0.60,0.02}{#1}}
\newcommand{\WarningTok}[1]{\textcolor[rgb]{0.56,0.35,0.01}{\textbf{\textit{#1}}}}
\usepackage{graphicx}
\makeatletter
\def\maxwidth{\ifdim\Gin@nat@width>\linewidth\linewidth\else\Gin@nat@width\fi}
\def\maxheight{\ifdim\Gin@nat@height>\textheight\textheight\else\Gin@nat@height\fi}
\makeatother
% Scale images if necessary, so that they will not overflow the page
% margins by default, and it is still possible to overwrite the defaults
% using explicit options in \includegraphics[width, height, ...]{}
\setkeys{Gin}{width=\maxwidth,height=\maxheight,keepaspectratio}
% Set default figure placement to htbp
\makeatletter
\def\fps@figure{htbp}
\makeatother
\setlength{\emergencystretch}{3em} % prevent overfull lines
\providecommand{\tightlist}{%
  \setlength{\itemsep}{0pt}\setlength{\parskip}{0pt}}
\setcounter{secnumdepth}{-\maxdimen} % remove section numbering
\ifluatex
  \usepackage{selnolig}  % disable illegal ligatures
\fi

\title{Example Presentation}
\author{}
\date{\vspace{-2.5em}}

\begin{document}
\frame{\titlepage}

\begin{frame}[fragile]{Overview}
\protect\hypertarget{overview}{}
This document has code embedded throughout. In the next section we will
create a visualization using the already loaded dataset
\texttt{cryptodata}:

\begin{Shaded}
\begin{Highlighting}[]
\FunctionTok{datatable}\NormalTok{(eth\_data, }\AttributeTok{rownames =} \ConstantTok{FALSE}\NormalTok{, }
          \FunctionTok{options}\NormalTok{(}\FunctionTok{list}\NormalTok{(}\AttributeTok{lengthMenu =} \FunctionTok{c}\NormalTok{(}\DecValTok{4}\NormalTok{, }\DecValTok{5}\NormalTok{, }\DecValTok{6}\NormalTok{))))}
\end{Highlighting}
\end{Shaded}

\includegraphics{beamer_files/figure-beamer/unnamed-chunk-1-1.pdf}
\end{frame}

\begin{frame}{Price Chart}
\protect\hypertarget{price-chart}{}
\includegraphics{beamer_files/figure-beamer/unnamed-chunk-2-1.pdf}
\end{frame}

\begin{frame}{Interactive Chart}
\protect\hypertarget{interactive-chart}{}
\includegraphics{beamer_files/figure-beamer/unnamed-chunk-3-1.pdf}
\end{frame}

\begin{frame}[fragile]{Python Code Example}
\protect\hypertarget{python-code-example}{}
\begin{Shaded}
\begin{Highlighting}[]
\ImportTok{import}\NormalTok{ pandas }\ImportTok{as}\NormalTok{ pd}
\CommentTok{\# Create the Python object from R}
\NormalTok{df }\OperatorTok{=}\NormalTok{ r.cryptodata}
\CommentTok{\# Show the new Python dataframe}
\NormalTok{df}
\end{Highlighting}
\end{Shaded}

\begin{verbatim}
##         pair symbol  ask_1_price       date_time_utc
## 0     BTCUSD    BTC    18037.560 2020-12-12 00:00:00
## 1     ETHUSD    ETH      544.422 2020-12-12 00:00:01
## 2     BTCUSD    BTC    18338.710 2020-12-12 01:00:00
## 3     ETHUSD    ETH      555.184 2020-12-12 01:00:01
## 4     BTCUSD    BTC    18283.790 2020-12-12 02:00:00
## ...      ...    ...          ...                 ...
## 5077  BTCUSD    BTC    11847.080 2020-08-10 21:03:49
## 5078  BTCUSD    BTC    11819.920 2020-08-10 22:03:49
## 5079  BTCUSD    BTC    11804.900 2020-08-10 23:03:54
## 5080  BTCUSD    BTC    10686.880                 NaT
## 5081  ETHUSD    ETH      357.844                 NaT
## 
## [5082 rows x 4 columns]
\end{verbatim}
\end{frame}

\begin{frame}[fragile]{One more Python example}
\protect\hypertarget{one-more-python-example}{}
Press on \texttt{w} on your keyboard to make the presentation wider.
Press \texttt{f} to fullscreen.

\begin{Shaded}
\begin{Highlighting}[]
\ImportTok{import}\NormalTok{ numpy }\ImportTok{as}\NormalTok{ np}
\CommentTok{\# Create a new field based on the ask\_1\_price value:}
\NormalTok{df[}\StringTok{\textquotesingle{}price\_percentile\textquotesingle{}}\NormalTok{] }\OperatorTok{=}\NormalTok{ np.where(df[}\StringTok{\textquotesingle{}ask\_1\_price\textquotesingle{}}\NormalTok{] }\OperatorTok{\textgreater{}}\NormalTok{ np.percentile(df[}\StringTok{\textquotesingle{}ask\_1\_price\textquotesingle{}}\NormalTok{], }\DecValTok{50}\NormalTok{),}
                            \StringTok{\textquotesingle{}upper 50th percentile of prices\textquotesingle{}}\NormalTok{, }
                            \StringTok{\textquotesingle{}lower 50th percentile of prices\textquotesingle{}}\NormalTok{)}
\CommentTok{\# Show modified dataframe:}
\NormalTok{df[[}\StringTok{\textquotesingle{}symbol\textquotesingle{}}\NormalTok{, }\StringTok{\textquotesingle{}ask\_1\_price\textquotesingle{}}\NormalTok{, }\StringTok{\textquotesingle{}price\_percentile\textquotesingle{}}\NormalTok{]]}
\end{Highlighting}
\end{Shaded}

\begin{verbatim}
##      symbol  ask_1_price                 price_percentile
## 0       BTC    18037.560  upper 50th percentile of prices
## 1       ETH      544.422  lower 50th percentile of prices
## 2       BTC    18338.710  upper 50th percentile of prices
## 3       ETH      555.184  lower 50th percentile of prices
## 4       BTC    18283.790  upper 50th percentile of prices
## ...     ...          ...                              ...
## 5077    BTC    11847.080  upper 50th percentile of prices
## 5078    BTC    11819.920  upper 50th percentile of prices
## 5079    BTC    11804.900  upper 50th percentile of prices
## 5080    BTC    10686.880  upper 50th percentile of prices
## 5081    ETH      357.844  lower 50th percentile of prices
## 
## [5082 rows x 3 columns]
\end{verbatim}
\end{frame}

\begin{frame}[fragile]{\href{https://r-markdown-gallery.org}{Back to
Gallery}}
\protect\hypertarget{back-to-gallery}{}
\begin{Shaded}
\begin{Highlighting}[]
\NormalTok{knitr}\SpecialCharTok{::}\FunctionTok{include\_url}\NormalTok{(}\StringTok{"https://r{-}markdown{-}gallery.org"}\NormalTok{)}
\end{Highlighting}
\end{Shaded}

\href{https://r-markdown-gallery.org}{\includegraphics{beamer_files/figure-beamer/unnamed-chunk-6-1.pdf}}
\end{frame}

\end{document}

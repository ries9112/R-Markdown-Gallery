\documentclass[]{tufte-handout}

% ams
\usepackage{amssymb,amsmath}

\usepackage{ifxetex,ifluatex}
\usepackage{fixltx2e} % provides \textsubscript
\ifnum 0\ifxetex 1\fi\ifluatex 1\fi=0 % if pdftex
  \usepackage[T1]{fontenc}
  \usepackage[utf8]{inputenc}
\else % if luatex or xelatex
  \makeatletter
  \@ifpackageloaded{fontspec}{}{\usepackage{fontspec}}
  \makeatother
  \defaultfontfeatures{Ligatures=TeX,Scale=MatchLowercase}
  \makeatletter
  \@ifpackageloaded{soul}{
     \renewcommand\allcapsspacing[1]{{\addfontfeature{LetterSpace=15}#1}}
     \renewcommand\smallcapsspacing[1]{{\addfontfeature{LetterSpace=10}#1}}
   }{}
  \makeatother

\fi

% graphix
\usepackage{graphicx}
\setkeys{Gin}{width=\linewidth,totalheight=\textheight,keepaspectratio}

% booktabs
\usepackage{booktabs}

% url
\usepackage{url}

% hyperref
\usepackage{hyperref}

% units.
\usepackage{units}


\setcounter{secnumdepth}{-1}

% citations


% pandoc syntax highlighting
\usepackage{color}
\usepackage{fancyvrb}
\newcommand{\VerbBar}{|}
\newcommand{\VERB}{\Verb[commandchars=\\\{\}]}
\DefineVerbatimEnvironment{Highlighting}{Verbatim}{commandchars=\\\{\}}
% Add ',fontsize=\small' for more characters per line
\newenvironment{Shaded}{}{}
\newcommand{\AlertTok}[1]{\textcolor[rgb]{1.00,0.00,0.00}{\textbf{#1}}}
\newcommand{\AnnotationTok}[1]{\textcolor[rgb]{0.38,0.63,0.69}{\textbf{\textit{#1}}}}
\newcommand{\AttributeTok}[1]{\textcolor[rgb]{0.49,0.56,0.16}{#1}}
\newcommand{\BaseNTok}[1]{\textcolor[rgb]{0.25,0.63,0.44}{#1}}
\newcommand{\BuiltInTok}[1]{#1}
\newcommand{\CharTok}[1]{\textcolor[rgb]{0.25,0.44,0.63}{#1}}
\newcommand{\CommentTok}[1]{\textcolor[rgb]{0.38,0.63,0.69}{\textit{#1}}}
\newcommand{\CommentVarTok}[1]{\textcolor[rgb]{0.38,0.63,0.69}{\textbf{\textit{#1}}}}
\newcommand{\ConstantTok}[1]{\textcolor[rgb]{0.53,0.00,0.00}{#1}}
\newcommand{\ControlFlowTok}[1]{\textcolor[rgb]{0.00,0.44,0.13}{\textbf{#1}}}
\newcommand{\DataTypeTok}[1]{\textcolor[rgb]{0.56,0.13,0.00}{#1}}
\newcommand{\DecValTok}[1]{\textcolor[rgb]{0.25,0.63,0.44}{#1}}
\newcommand{\DocumentationTok}[1]{\textcolor[rgb]{0.73,0.13,0.13}{\textit{#1}}}
\newcommand{\ErrorTok}[1]{\textcolor[rgb]{1.00,0.00,0.00}{\textbf{#1}}}
\newcommand{\ExtensionTok}[1]{#1}
\newcommand{\FloatTok}[1]{\textcolor[rgb]{0.25,0.63,0.44}{#1}}
\newcommand{\FunctionTok}[1]{\textcolor[rgb]{0.02,0.16,0.49}{#1}}
\newcommand{\ImportTok}[1]{#1}
\newcommand{\InformationTok}[1]{\textcolor[rgb]{0.38,0.63,0.69}{\textbf{\textit{#1}}}}
\newcommand{\KeywordTok}[1]{\textcolor[rgb]{0.00,0.44,0.13}{\textbf{#1}}}
\newcommand{\NormalTok}[1]{#1}
\newcommand{\OperatorTok}[1]{\textcolor[rgb]{0.40,0.40,0.40}{#1}}
\newcommand{\OtherTok}[1]{\textcolor[rgb]{0.00,0.44,0.13}{#1}}
\newcommand{\PreprocessorTok}[1]{\textcolor[rgb]{0.74,0.48,0.00}{#1}}
\newcommand{\RegionMarkerTok}[1]{#1}
\newcommand{\SpecialCharTok}[1]{\textcolor[rgb]{0.25,0.44,0.63}{#1}}
\newcommand{\SpecialStringTok}[1]{\textcolor[rgb]{0.73,0.40,0.53}{#1}}
\newcommand{\StringTok}[1]{\textcolor[rgb]{0.25,0.44,0.63}{#1}}
\newcommand{\VariableTok}[1]{\textcolor[rgb]{0.10,0.09,0.49}{#1}}
\newcommand{\VerbatimStringTok}[1]{\textcolor[rgb]{0.25,0.44,0.63}{#1}}
\newcommand{\WarningTok}[1]{\textcolor[rgb]{0.38,0.63,0.69}{\textbf{\textit{#1}}}}

% longtable

% multiplecol
\usepackage{multicol}

% strikeout
\usepackage[normalem]{ulem}

% morefloats
\usepackage{morefloats}


% tightlist macro required by pandoc >= 1.14
\providecommand{\tightlist}{%
  \setlength{\itemsep}{0pt}\setlength{\parskip}{0pt}}

% title / author / date
\title{R Markdown Output}
\date{}


\begin{document}

\maketitle




\hypertarget{overview}{%
\section{Overview}\label{overview}}

This document has code embedded throughout. In the next section we will
create a visualization using the already loaded dataset
\texttt{eth\_data}:

\begin{Shaded}
\begin{Highlighting}[]
\KeywordTok{datatable}\NormalTok{(eth_data)}
\end{Highlighting}
\end{Shaded}

\hypertarget{htmlwidget-402b29d077ba7b190496}{}

\hypertarget{price-chart---ethereum}{%
\section{Price Chart - Ethereum}\label{price-chart---ethereum}}

\includegraphics{tufte_handout_files/figure-latex/unnamed-chunk-2-1}

\hypertarget{python-code-example}{%
\section{Python Code Example}\label{python-code-example}}

\begin{Shaded}
\begin{Highlighting}[]
\ImportTok{import}\NormalTok{ pandas }\ImportTok{as}\NormalTok{ pd}
\CommentTok{# Create the Python object from R}
\NormalTok{df }\OperatorTok{=}\NormalTok{ r.cryptodata}
\CommentTok{# Show the new Python dataframe}
\NormalTok{df}
\end{Highlighting}
\end{Shaded}

\begin{verbatim}
##          pair symbol  ask_1_price       date_time_utc
## 0      ETHUSD    ETH     1982.230 2021-07-13 06:00:01
## 1      BTCUSD    BTC    32825.540 2021-07-13 06:00:00
## 2      ETHUSD    ETH     2024.858 2021-07-13 05:00:01
## 3      BTCUSD    BTC    33140.010 2021-07-13 05:00:00
## 4      ETHUSD    ETH     2030.624 2021-07-13 04:00:01
## ...       ...    ...          ...                 ...
## 15172  BTCUSD    BTC    11747.010 2020-08-14 04:03:56
## 15173  BTCUSD    BTC    11722.060 2020-08-14 03:03:55
## 15174  BTCUSD    BTC    11761.120 2020-08-14 02:04:04
## 15175  BTCUSD    BTC    11719.280 2020-08-14 01:03:54
## 15176  BTCUSD    BTC    11827.080 2020-08-14 00:03:56
## 
## [15177 rows x 4 columns]
\end{verbatim}

\hypertarget{one-more-python-example}{%
\section{One more Python example}\label{one-more-python-example}}

The code below creates a new column \texttt{price\_percentile} that
specifies if the price for the row was in the upper or lower 50th
percentile of prices (BTC should be upper and ETH lower):

\begin{Shaded}
\begin{Highlighting}[]
\ImportTok{import}\NormalTok{ numpy }\ImportTok{as}\NormalTok{ np}
\CommentTok{# Create a new column based on the ask_1_price value:}
\NormalTok{df[}\StringTok{'price_percentile'}\NormalTok{] }\OperatorTok{=}\NormalTok{ np.where(df[}\StringTok{'ask_1_price'}\NormalTok{] }\OperatorTok{>} 
\NormalTok{                                  np.percentile(df[}\StringTok{'ask_1_price'}\NormalTok{], }\DecValTok{50}\NormalTok{),}
                            \StringTok{'upper 50th percentile of prices'}\NormalTok{, }
                            \StringTok{'lower 50th percentile of prices'}\NormalTok{)}
\CommentTok{# Show modified dataframe:}
\NormalTok{df[[}\StringTok{'symbol'}\NormalTok{, }\StringTok{'ask_1_price'}\NormalTok{, }\StringTok{'price_percentile'}\NormalTok{]]}
\end{Highlighting}
\end{Shaded}

\begin{verbatim}
##       symbol  ask_1_price                 price_percentile
## 0        ETH     1982.230  lower 50th percentile of prices
## 1        BTC    32825.540  upper 50th percentile of prices
## 2        ETH     2024.858  lower 50th percentile of prices
## 3        BTC    33140.010  upper 50th percentile of prices
## 4        ETH     2030.624  lower 50th percentile of prices
## ...      ...          ...                              ...
## 15172    BTC    11747.010  upper 50th percentile of prices
## 15173    BTC    11722.060  upper 50th percentile of prices
## 15174    BTC    11761.120  upper 50th percentile of prices
## 15175    BTC    11719.280  upper 50th percentile of prices
## 15176    BTC    11827.080  upper 50th percentile of prices
## 
## [15177 rows x 3 columns]
\end{verbatim}



\end{document}
